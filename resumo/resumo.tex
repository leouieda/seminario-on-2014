\documentclass[12pt]{letter}

\usepackage{geometry}
\geometry{
	a4paper,
	total={210mm,297mm},
	left=20mm,
	right=20mm,
	top=25mm,
	bottom=20mm,
}

\usepackage[brazilian]{babel}
\usepackage[utf8]{inputenc}
\usepackage[T1]{fontenc}
\usepackage{fancyhdr}
\usepackage{graphicx}

\renewcommand{\headrulewidth}{0pt}
\fancyhead[L]{
	\includegraphics[width=5cm]{logo_on}
}

\begin{document}

\thispagestyle{fancy}

\begin{flushleft}

\textbf{Modelagem e inversão em coordenadas esféricas na gravimetria}\\
\line(1,0){450}
\\\textbf{Estudante:} Leonardo Uieda\\
\line(1,0){450}
\\\textbf{Orientador:} Valéria Cristina Ferreira Barbosa\\
\line(1,0){450}
\\\textbf{Nível:} Doutorado\\
\line(1,0){450}
\\\textbf{Período Previsto de Bolsa de Estudos:} 2011 - 2015\\
\line(1,0){450}
\\\textbf{Período a que se Refere o Relatório:} 2014\\
\line(1,0){450}

\textbf{Resumo}
\\
\vspace{10 mm}

Existem diversos métodos de inversão linear 3D.

A maioria discretiza a Terra em prismas retangulares retos homogêneos.

O problema inverso consiste em estimar a distribuição de densidade dos prismas
que ajuste a anomalia da gravidade medida.

Levar em conta a curvatura da Terra é desejável em estudos de modelagem em
escala regional (centenas de quilômetros) ou global.

Uma abordagem para isso é discretizar a Terra em prismas esféricos, também
chamados de tesseroides.

Assim, métodos existentes de inversão utilizando prismas podem ser adaptados
para usar tesseroides.

Esta adaptação requer:
um método para calcular o efeito gravitacional de um tesseroide com precisão
e um software que implemente tanto o cálculo direto quanto o método de inversão.


O potencial gravitacional de um tesseroide e suas derivadas não possuem
soluções analíticas.

Foram propostos dois métodos para a integração numérica: a expansão do
integrando em série de Taylor e a Quadratura Gauss-Legendre.

A Quadratura Gauss-Legendre é o método de integração numérica que escolhemos.

A quadratura consiste em aproximar a integral por uma soma ponderada do efeito
gravitacional de N massas pontuais.

Essas massas pontuais são localizadas em pontos no interior do tesseroide que
correspondem a raízes de um polinômio de Legendre de grau N.

O número de massas utilizando (N) é chamado de ``ordem'' da quadratura.

A acurácia da integração numérica depende do número de massas utilizadas e da
distância do tesseroide ao ponto de observação.

Quanto mais próximo estiver o ponto de observação, mais massas são necessárias.

Este comportamento pode ser justificado do ponto de vista da teoria da
amostragem.

É conhecido em métodos potenciais que quanto mais próximo for o ponto de observação, maior é a frequência espacial
(número de onda) do campo potencial.

Consequentemente, na integração numérica são necessários mais pontos para discretizar o integrando sem
que ocorra falseamento.



Um método proposto na literatura para aumentar o número de massas pontuais sem
aumentar a ordem da quadratura é dividir o tesseroide em tesseroides menores.

O campo calculado é a soma dos campos dos tesseroides menores, que é calculado com um número fixo de massas.

Esta divisão se repete para cada tesseroide menor até que ponto de observação
esteja distante o suficiente em relação ao tamanho do tesseroide.

Desta forma, o número de divisões, e consequentemente o número de massas
pontuais, é maior nas partes do tesseroide próximas do ponto de observação.

Podemos dizer que a acurácia da integração numérica depende da razão entre a
distância até o ponto de observação e o tamanho do tesseroide.

Trabalhos anteriores sugerem que esta razão deve ser próxima de 1 (um).

Porém, atualmente não há uma análise que relacione quantitativamente
a maior razão distância/tamanho permitida com o erro cometido na integração.



Neste trabalho, investigamos qual é o erro cometido na integração através da
divisão dos tesseroides.

Utilizamos como referência o modelo de uma casca esférica, para a qual existe
uma solução analítica.

Esta casca é discretizada em tesseroides.

Comparamos o potencial gravitacional e suas primeiras e segundas derivadas
gerados pelo modelo de tesseroides com o valor equivalente para a casca esférica.

Este cálculo foi feito para diversos valores da máxima razão distância/tamanho
permitida.

Fizemos um gráfico do erro cometido na integração pela razão
distância/tamanho.

Desta forma, determinamos qual é o valor da razão distância/tamanho que
proporciona um erro máximo aceitável.

Esta análise pode ser feita para o potencial gravitacional e suas derivadas.



Utilizamos os resultados acima na modelagem direta para adaptar a coordenadas
esféricas o método de inversão de ``plantação de densidades''.

Este método utiliza um algoritmo de busca sistemática para construir a solução
do problema inverso.

O algoritmo adiciona tesseroides em torno de um conjunto inicial de tesseroides
``sementes'' fornecidos pelo interprete.

Desta maneira, a solução cresce em torno das sementes até que os dados preditos
pela solução ajustem os dados observados.

A maior vantagem do método de plantação é sua eficiência computacional.

O algoritmo não requer a solução de sistemas lineares e matriz de sensibilidade
pode ser calculada de forma eficiente.

Realizamos testes com dados sintéticos tanto da anomalia da gravidade como
do tensor gradiente da gravidade.




A modelagem direta com tesseroides e a inversão por algoritmo de plantação
foram implementadas no software Fatiando a Terra (www.fatiando.org).

Recentemente, desenvolvemos um novo módulo chamado \texttt{fatiando.inversion}.

Este módulo automatiza grande parte da solução de um problema inverso.

Para implementar um problema novo, o usuário necessita somente desenvolver a
solução do problema direto e a criação da matriz de sensibilidade.

Uma vez feitas essas duas etapas, o usuário tem acesso automático a diversos
métodos de optimização e regularização.

Desta forma é possível reutilizar as rotinas de modelagem direta implementadas
no Fatiando a Terra para desenvolver novos métodos de inversão e adaptar
adaptar métodos existentes a novas parametrizações.


Em conclusão,
determinamos um valor ideal para a razão entre a distância ao ponto observação
e o tamanho de um tesseroide através da comparação com uma casca esférica.

Esta razão é utilizada na modelagem direta em um algoritmo de divisão automática dos tesseroides.

Observamos que o valor ideal para a razão distância/tamanho é diferente para o
potencial e suas derivadas segundas.

Esses resultados foram utilizados na adaptação do método de inversão por
plantação de densidades para coordenadas esféricas.

Esta adaptação foi testada em dados sintéticos que simulam diferentes ambientes
geológicos.

Para a aplicação em dados reais, serão utilizados dados de modelos
geopotenciais.

Também investigaremos a utilização dos dados de gravimetria da gravidade
fornecidos pelo satélite GOCE.



\vspace{10 mm}
\textbf{Palavras-Chave:}
Inversão. Gravimetria. Coordenadas esféricas. Tesseroide.

\end{flushleft}
\end{document}
